\documentclass[parskip=full,11pt,twoside]{scrartcl}

\usepackage[l2tabu,orthodox]{nag}
\usepackage[utf8]{inputenc}

\title{Wavelength $\bm{\lambda}$-IDE}
\author{Muhammet Guemues, Markus Himmel, Marc Huisinga,\\Philip Klemens, Julia Schmid, Jean-Pierre von der Heydt}

% section numbers in margins:
\renewcommand\sectionlinesformat[4]{\makebox[0pt][r]{#3}#4}

% header & footer
\usepackage{scrlayer-scrpage}
\lofoot{\today}
\refoot{\today}
\pagestyle{scrheadings}

%\usepackage[sfdefault,light]{roboto}
\usepackage[T1]{fontenc}
\usepackage[ngerman]{babel}
\usepackage{datetime2}
\usepackage{hyperref}
\usepackage[nameinlink]{cleveref}
\crefname{figure}{Abb}{Abb}
\usepackage[section]{placeins}
\usepackage{xcolor}
\usepackage{graphicx}
\hypersetup{
	pdftitle={Pflichtenheft},
	bookmarks=true,
}
\usepackage{csquotes}

\usepackage{amsmath}
\usepackage{bm}
\usepackage{mathtools}

\usepackage{glossaries}
\GlsSetQuote{+}
%
\usepackage{pkg/pflichtenheft}

\MakeOuterQuote{"}

\makeglossaries

\newglossaryentry{lk}
{ 
	name=$\lambda$-Kalkül,
	description={Der $\lambda$-Kalkül ist eine formale Sprache, die, basierend auf Funktionsdefinitionen, eine systematische Untersuchung von Funktionen zulässt}
}
\newglossaryentry{lapp}
{ 
	name=$\lambda$-Applikation,
	plural=$\lambda$-Applikationen,
	description={Applikationen stellen die Anwendung eines \gls{lt}s auf einen weiteren \gls{lt} dar. Sie haben im Allgemeinen die Form: \emph{<\gls{lt}1> <\gls{lt}2>}, wobei die spitzen Klammern hier nur der Übersicht wegen eingefügt wurden und im eigentlichen 			    Ausdruck nicht vorkommen}
}
\newglossaryentry{labs}
{ 
	name=$\lambda$-Abstraktion,
	plural=$\lambda$-Abstraktionen,
	description={Abstraktionen stellen eine Funktionsdefinition einer Funktion in einer Variablen dar. Abstraktionen haben die Form \emph{$\lambda$} <\gls{var}>.<\gls{lt}>, wobei die Ausdrücke in den spitzen Klammern inklusive der spitzen Klammern durch die entsprechenden Objekte zu ersetzen sind}
}
\newglossaryentry{lt}
{
	name=$\lambda$-Term,
	plural=$\lambda$-Terme,
	description={ Terme im \gls{lk} sind entweder \emph{\glspl{lapp}} oder \emph{\glspl{labs}} oder \emph{\glspl{var}} }
}
\newglossaryentry{var}
{ 
	name=Variable,
	plural=Variablen,
	description={Variablen sind Platzhalter für konkrete Werte. Als Variablen zugelassen sind beliebige Zeichen und Zeichenkombinationen bis auf die folgenden:
	\begin{tabular} {| l | c | r |}
	\hline
	 $\lambda$ & . & \textbackslash \\
	 \hline
	\end{tabular}
	}
}
\newglossaryentry{alpha} 
{
	name=$\alpha$-Äquivalenz,
	description={Die $\alpha$-Äquivalenz zweier \glspl{lt} bedeutet, dass einer der beiden Terme durch Umbenennung der Variablen in den anderen Term überführt werden kann}
}
\newglossaryentry{beta}
{
	name=$\beta$-Reduktion,
	description={Die $\beta$-Reduktion entspricht einer Funktionsauswertung. Sie ist nur auf \glspl{redex} anwendbar. Formal gilt für eine Variable $V$ und \glspl{lt} $T_1$ und $T_2$ die \gls{subs}: $(\lambda V.T_1) T_2 \stackrel{\beta}{\Rightarrow} 			T_1[T_2/V]$}
}
\newglossaryentry{redex} 
{
	name=Redex,
	plural=Redexe,
	description={Redexe (Reducible Expressions) sind Ausdrücke der Form \emph{($\lambda$.\gls{lt}1) \gls{lt}2}}
}
\newglossaryentry{fv} 
{
	name=freie Variable,
	plural=freie Variablen,
	description={Eine Variable in einem \gls{lt} heißt \emph{frei}, wenn sie kein Parameter einer \gls{labs} ist}
}
\newglossaryentry{gv} 
{
	name=gebundene Variable,
	plural=gebundene Variablen,
	description={Eine Variable in einem \gls{lt} heißt \emph{gebunden}, wenn sie Parameter einer \gls{labs} ist}
}
\newglossaryentry{subs}
{
	name=Substitution,
	plural=Substitutionen,
	description={Bei einer Substitution einer \glspl{var} $y$ in einer \gls{labs} $\lambda x.T_1$, wobei $T_1$ ein \gls{lt} ist,  werden alle freien Vorkommen von $x$ in $T_1$ durch $y$ ersetzt und der Parameter in y umbenannt. 
			In Formeln: $\lambda x.T_1 \stackrel{\text{Substitution mit $y$}}{\Longrightarrow}\lambda y.[T_1/y]$}
}
\newglossaryentry{aws}
{
	name=Auswertungsstrategie,
	plural=Auswertungsstrategien,
	description={Eine Auswertungsstrategie ist ein Algorithmus, der bestimmt, wie ein \gls{lt} ausgewertet wird. Dabei kann es vorkommen, dass eine Auswertungsstrategie bei der Auswertung eines \gls{lt} in eine Endlosschleife gerät, während eine andere 				Strategie terminiert}
}
\newglossaryentry{vbr}
{
	name=volle $\beta$-Reduktion,
	description={Die volle $\beta$-Reduktion ist eine \gls{aws}, bei der jeder \gls{redex} jederzeit $\beta$-reduziert werden kann. Der Nutzer bekommt bei dieser \gls{aws} die Möglichkeit die entsprechende Auswertungsreihenfolge selbst zu bestimmen}
}
\newglossaryentry{nr}
{
	name=normale Reduktionsordnung,
	description={Die normale Reduktionsordnung ist eine \gls{aws}, bei der immer der linkeste äußerste \gls{redex} zuerst $\beta$-reduziert wird}
}
\newglossaryentry{cbn}
{
	name=Call by Name,
	description={Call by Name ist eine \gls{aws}, bei der der linkeste äußerste \gls{redex}, der nicht von einem $\lambda$ umgeben ist, zuerst $\beta$-reduziert wird}
}
\newglossaryentry{cbv}
{
	name=Call by Value,
	description={Call by Value ist eine \gls{aws}, bei der der linkeste äußerste \gls{redex}, der nicht von einem $\lambda$ umgeben ist und dessen Argument ein konkreter Wert ist, zuerst $\beta$-reduziert wird.}
}
\newglossaryentry{yc} 
{
	name=Y-Kombinator,
	description={Formal dient der Y-Kombinator dazu, \gls{rec} zu ermöglichen. Er ist definiert als: $Y \coloneqq \lambda f.(\lambda x.f(x\,x))(\lambda x.f(x\,x))$}
}	
\newglossaryentry{rec}
{
	name=Rekursion,
	description={Eine Funktion heißt rekursiv, wenn sie sich selbst aufruft. Der Aufruf wird dann als Rekursion bezeichnet}
}
\newglossaryentry{brow}
{
	name=kompatibler Browser,
	plural=kompatible Browser,
	description={Ein Browser, der in Abschnitt (hier ein Label) für unterstützt erklärt wurde}
}
\newglossaryentry{ao}
{
	name=Applicative Order,
	description={Eine \gls{aws}, bei der der rechteste innerste Redex zuerst $\beta$-reduziert wird}
}
\newglossaryentry{mfe}
{
	name=Mehrfacheinrückung,
	plural=Mehrfacheinrückungen,
	description={Mehrfacheinrückungen ergänzen die \glspl{st} um verschiedene Einrückungsebenen}
}
\newglossaryentry{st}
{
	name=Smart Tab,
	plural=Smart Tabs,
	description={Bei einem Zeilenwechsel wird die neue Zeile genauso eingerückt wie die alte Zeile}
}
%%% Beginn Glossareinträge

\newglossaryentry{vls}
{
	name={vereinfachte $\lambda$-Syntax},
	description={Variante der herkömmlichen Syntax des \glslink{lk}{$\lambda$-Kalküls},
	bei der das $\lambda$-Symbol durch ein \texttt{\textbackslash}-Symbol ersetzt wird}
}

% Syntax:
%\newglossaryentry{label}
%{
%	name=Name,
%	plural=Namen,
%	description={Beschreibung}
%}

% Verwendung der Glossareinträge:
% Normalerweise \gls{label}
% Am Anfang des Satzes \Gls{label}
% Bei Plural: \glspl{label}
% Bei Plural am Anfang des Satzes: \Glspl{label}
% Falls nichts davon passt: \glslink{label}{Anderer Text}

%%% Ende Glossareinträge

\begin{document}
\maketitle

\section{Einleitung}


\pagebreak
\section{Kriterien}
% Diese Section sollte kurz und knapp "für Manager" sein
% und auf eine Seite passen.

\subsection{Muss}

\criterium{Eingabe von \glslink{lt}{$\lambda$-Termen}}{crt:input}
\glspl{lt} können in Form der \glslink{vls}{vereinfachten $\lambda$-Syntax}
mit der Tastatur in die Software eingegeben werden.

\criterium{Reduktion von \glslink{lt}{$\lambda$-Termen}}{crt:reduce}
Eingegebene \glspl{lt} können, sofern möglich, mit Hilfe der \glslink{nr}{Normal-Reduktionsordnung} vollständig
reduziert werden. Die so bestimmte Normalform kann in \glslink{vls}{vereinfachter $\lambda$-Syntax}
ausgegeben werden.

\criterium{Fehlermeldung bei invalider Eingabe}{crt:error}
Beim dem Versuch, einen syntaktisch inkorrekten \gls{lt} reduzieren zu lassen, wird eine
Fehlermeldung ausgegeben.

\criterium{Abbruch der Ausführung}{ctr:cancel}
Der Benutzer kann die Reduktion eines \glslink{lt}{$\lambda$-Terms} abbrechen.

% Syntax:
% \criterium{Überschrift des Kriteriums}{crt:label}
% Beschreibung des Kriteriums

\subsection{Kann}
\criteriumOptional{Verfügbarkeit im Internet}{crt:webapp}
Die Anwendung kann ohne Installation in einem \glslink{brow}{kompatiblen Internetbrowser} ausgeführt werden.

\criteriumOptional{Weitere Reduktionsordnungen}{crt:reducestrats}
Die Anwendung kann neben der \glslink{nr}{Normal-Reduktionsordnung} auch \gls{cbn}, \gls{cbv}
und \gls{ao} als \gls{aws} verwenden.

\criteriumOptional{Weitere Ausgabeformate}{crt:output}
Die eingegebenen und reduzierten \glspl{lt} können neben der \glslink{vls}{vereinfachten $\lambda$-Syntax}
auch als Unicode-Text, \LaTeX-Quellcode, Haskell-Quellcode und Lisp-Quellcode formatiert und ausgegeben
werden.

\criteriumOptional{Darstellungsformate}{crt:display}
Die eingegebenen und reduzierten \glspl{lt} können innerhalb der Applikation in
\glslink{vls}{vereinfachter $\lambda$-Syntax}, Unicode-Darstellung und als Syntaxbaum
dargestellt werden.

\criteriumOptional{Erweiterte Fehlerdiagnostik}{crt:diag}
Im Falle syntaktischer Fehler werden dem Nutzer die genaue Position des Fehlers, die
Art des Fehlers sowie Behebungsmöglichkeiten angezeigt und die relevante Stelle im
Editor hervorgehoben.

\criteriumOptional{Intelligenter Editor}{crt:editor}
Der Editor zur Eingabe der \glspl{lt} unterstützt einen Wechsel der Schriftgröße,
\glspl{mfe} und \glspl{st}.

\criteriumOptional{Standardbibliothek}{crt:lib}
Es existiert eine Standardbibliothek, die Funktionen für den Umgang mit natürlichen
Zahlen, Listen und Tupeln und den Kombinatoren des SKI-Kalküls bereitstellt.

\criteriumOptional{Übungsaufgaben}{crt:ex}
Das Programm enthält ein Übungsaufgaben-System, welches neben Aufgaben in aufsteigendem
Schwierigkeitsgrad auch Einführungen und automatisierte Tests für die Lösungen des Nutzers
enthält.

\criteriumOptional{Ausgabe von Teilschritten}{crt:partial}
Die zum Erreichen der Normalform notwendigen Reduktionsschritte gemäß der ausgewählten
Reduktionsstrategie können im ausgewählten Ausgabeformat ausgegeben werden.

\criteriumOptional{Schritt-für-Schritt-Auswertung}{crt:steps}
Durch Klick kann ein bestimmter Teilausdruck ausgewertet werden oder die ausgewählte
\gls{aws} einen einzelnen Schritt machen. Diese Auswertungsschritte können rückgängig
gemacht werden.

\criteriumOptional{Permalinks}{crt:perma}
Es kann ein Link generiert werden, der beim Aufrufen des gesamten Zustand
der Sitzung widerherstellt.

% Syntax:
% \criteriumOptional{Überschrift des Kriteriums}{crt:label}
% Beschreibung des Kriteriums


\subsection{Abgrenzung}

% Syntax:
% \criteriumNot{Überschrift der Abgrenzung}{ctr:label}
\criteriumNot{Nutzung über Mobilgeräte}{crt:mobile}
Die Anwendung soll nicht auf mobilen Endgeräten eingesetzt werden.

\pagebreak
%%%%%%%%%%%%%%
\section{Produkteinsatz}

\section{Produktumgebung}

%%%%%%%%%%%
\section{Funktionale Anforderungen}

% Syntax:
% \functionality{Überschrift der FA}{fnc:label}
% \fulfills{crt:label1}
% \fulfills{crt:label2}
% Beschreibung der FA

%%%%%%%%%%%
\section{Nicht-Funktionale Anforderungen}

% Syntax:
% \nonFunctionality{Überschrift der NA}{nfc:label}
% Brschreibung der NA

%%%%%%%%%%%
\section{Tests}

% Syntax:
% \test{Überschrift des Tests}{tst:label}
% \tests{fnc:label}
% \tests{nfc:label}
%
% \teststep{Stand}
% {Aktion}
% {Reaktion}
%
% \teststep{Stand}
% {Aktion}
% {Reaktion}
%
% ...

%%%%%%%%%%%%%
\pagebreak
\appendix

\section{Seitenentwürfe}

% Hier ganz normale figures mit Bildern

\section{Glossar}

\glsaddall
\printglossaries

\end{document}
