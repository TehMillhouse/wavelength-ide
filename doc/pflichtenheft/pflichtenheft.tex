\documentclass[parskip=full,11pt,twoside]{scrartcl}

\usepackage[l2tabu,orthodox]{nag}
\usepackage[utf8]{inputenc}

\title{Wavelength $\bm{\lambda}$-IDE}
\author{Muhammet Guemues, Markus Himmel, Marc Huisinga,\\Philip Klemens, Julia Schmid, Jean-Pierre von der Heydt}

% section numbers in margins:
\renewcommand\sectionlinesformat[4]{\makebox[0pt][r]{#3}#4}

% header & footer
\usepackage{scrlayer-scrpage}
\lofoot{\today}
\refoot{\today}
\pagestyle{scrheadings}

\usepackage[sfdefault,light]{roboto}
\usepackage[T1]{fontenc}
\usepackage[ngerman]{babel}
\usepackage{datetime2}
\usepackage{hyperref}
\usepackage[nameinlink]{cleveref}
\crefname{figure}{Abb}{Abb}
\usepackage[section]{placeins}
\usepackage{xcolor}
\usepackage{graphicx}
\hypersetup{
	pdftitle={Pflichtenheft},
	bookmarks=true,
}
\usepackage{csquotes}

\usepackage{amsmath}
\usepackage{bm}

\usepackage{glossaries}
\GlsSetQuote{+}
%
\usepackage{pkg/pflichtenheft}

\MakeOuterQuote{"}

\makeglossaries

\newglossaryentry{lk}
{ 
	name=$\lambda$-Kalkül,
	description={Der $\lambda$-Kalkül ist eine formale Sprache, die, basierend auf Funktionsdefinitionen, eine systematische Untersuchung von Funktionen zulässt}
}
\newglossaryentry{lapp}
{ 
	name=$\lambda$-Applikation,
	plural=$\lambda$-Applikationen,
	description={Applikationen stellen die Anwendung eines \gls{lt}s auf einen weiteren \gls{lt} dar. Sie haben im Allgemeinen die Form : \emph{<\gls{lt}1> <\gls{lt}2>}, wobei die spitzen Klammern hier nur der Übersicht wegen eingefügt wurden und im eigentlichen 			    Ausdruck nicht vorkommen}
}
\newglossaryentry{labs}
{ 
	name=$\lambda$-Abstraktion,
	plural=$\lambda$-Abstraktionen,
	description={Abstraktionen stellen eine Funktionsdefinition dar, allerdings gilt die Einschränkung, dass diese nur einen Parameter haben dürfen. Die Funktion bekommt den Namen $\lambda$, gefolgt von einer Variablen, dann wird ein \emph{.} gesetzt und 
			    anschließend folgt ein \gls{lt}. Abstraktionen haben also die Form : \emph{$\lambda$} <\gls{var}>.<\gls{lt}>, wobei die spitzen Klammern nur der Übersicht dienen und im eigentlichen Ausdruck nicht vorkommen}
}
\newglossaryentry{lt}
{
	name=$\lambda$-Term,
	plural=$\lambda$-Terme,
	description={ Terme im \gls{lk} sind entweder \emph{\glspl{lapp}} oder \emph{\glspl{labs}} oder \emph{\glspl{var}} }
}
\newglossaryentry{var}
{ 
	name=Variable,
	plural=Variablen,
	description={Variablen sind Platzhalter für konkrete Werte. Als Variablen zugelassen sind beliebige Zeichen und Zeichenkombinationen bis auf die folgenden : 
	\begin{tabular} {| l | c | r |}
	\hline
	 $\lambda$ & . & \textbackslash \\
	 \hline
	\end{tabular}
	}	
}
\newglossaryentry{alpha} 
{
	name=$\alpha$-Äquivalenz,
	description={Die $\alpha$-Äquivalenz zweier \glspl{lt} bedeutet, dass einer der beiden Terme durch Umbenennung der Variablen in den anderen Term überführt werden kann}
}
\newglossaryentry{beta}
{
	name=$\beta$-Reduktion,
	description={Die $\beta$-Reduktion entspricht einer Funktionsauswertung. Diese ist aber nur auf \glspl{redex} anwendbar. Formal gilt für eine Variable $V$ und \glspl{lt} $T_1$ und $T_2$ die \gls{subs}: $(\lambda V.T_1) T_2 \stackrel{\beta}{\Rightarrow} 			T_1[T_2\textbackslash V]$}
}
\newglossaryentry{redex} 
{
	name=Redex,
	plural=Redexe,
	description={Redexe (Regular Expressions) sind Ausdrücke der Form : \emph{($\lambda$.\gls{lt}1) \gls{lt}2}}
}
\newglossaryentry{fv} 
{
	name=freie Variable,
	plural=freie Variablen,
	description={Eine Variable in einem \gls{lt} heißt \emph{frei}, wenn sie kein Parameter einer \gls{labs} ist.}
}
\newglossaryentry{gv} 
{
	name=gebundene Variable,
	plural=gebundene Variablen,
	description={Eine Variable in einem \gls{lt} heißt \emph{gebunden}, wenn sie Parameter einer \gls{labs} ist.}
}
\newglossaryentry{subs}
{
	name=Substitution,
	plural=Substitutionen,
	description={Bei einer Substitution einer \glspl{var} $y$ in einer \gls{labs} $\lambda x.T_1$, wobei $T_1$ ein \gls{lt} ist,  werden alle freien Vorkommen von x in $T_1$ mit y ersetzt und der Parameter in y umbenannt. 
			In Formeln : $\lambda x.T_1 \stackrel{Substitution mit y}{\Longrightarrow}\lambda y.[T_1\textbackslash y]$}
}
\newglossaryentry{aws}
{
	name=Auswertungsstrategie,
	plural=Auswertungsstrategien,
	description={Eine Auswertungsstrategie ist ein Algorithmus, der bestimmt, wie ein \gls{lt} ausgewertet wird. Dabei kann es vorkommen, dass eine Auswertungsstrategie bei der Auswertung eines \gls{lt} in eine Endlosschleife gerät, während eine andere 				Strategie terminiert. Die Anwendung stellt folgende Strategien bereit : \gls{nr}, \gls{cbn}, \gls{cbv}}
}
\newglossaryentry{vbr}
{
	name=volle $\beta$-Reduktion,
	description={Die volle $\beta$-Reduktion ist eine \gls{aws}, bei der jeder \gls{redex} jederzeit $\beta$-reduziert werden kann. Der Nutzer bekommt bei dieser \gls{aws} die Möglichkeit die entsprechende Auswertungsreihenfolge selbst zu bestimmen}
}
\newglossaryentry{nr}
{
	name=normale Reduktionsordnung,
	description={Die normale Reduktionsordnung ist eine \gls{aws}, bei der immer der linkeste äußerste \gls{redex} zuerst $\beta$-reduziert wird}
}
\newglossaryentry{cbn}
{
	name=Call by Name,
	description={Call by Name ist eine \gls{aws}, bei der der linkeste äußerste \gls{redex}, der nicht von einem $\lambda$ umgeben ist, zuerst $\beta$-reduziert wird}
}
\newglossaryentry{cbv}
{
	name=Call by Value,
	description={Call by Value ist eine \gls{aws}, bei der der linkeste äußerste \gls{redex}, der nicht von einem $\lambda$ umgeben ist und dessen Argument ein konkreter Wert ist, zuerst $\beta$-reduziert wird.}
}
\newglossaryentry{yc} 
{
	name=Y-Kombinator,
	description={Formal dient der Y-Kombinator dazu, \gls{rec} zu ermöglichen. Er ist definiert als : $Y := \lambda f.(\lambda x.f(x x))(\lambda x.f(x x))$}
}	
\newglossaryentry{rec}
{
	name=Rekursion,
	description={Eine Funktion heißt rekursiv, wenn sie sich selbst aufruft. Der Aufruf wird dann als Rekursion bezeichnet.}
}
\newglossaryentry{vls}
{
	name=vereinfachte $\lambda$-Syntax,
	description={Zur Vereinfachung der Benutzereingaben genügt es einen \emph{\textbackslash} statt eines \emph{$\lambda$} einzufügen}
}
\newglossaryentry{brow}
{
	name=kompatibler Browser,
	plural=kompatible Browser,
	description={Für folgende Browser kann Kompatibilität garantiert werden : Mozilla Firefox, Microsoft Internet Explorer (ab Version 8), Safari (ab Version 5), Chromium, Google Chrome und Opera} 
}
\newglossaryentry{ao}
{
	name=Applicative Order,
	description={Applicative Order ist eine \gls{aws}, bei der der rechteste innerste Redex zuerst $\beta$-reduziert wird}
}
\newglossaryentry{mfe}
{
	name=Mehrfacheinrückung,
	plural=Mehrfacheinrückungen,
	description={Mehrfacheinrückungen ergänzen die \glspl{st} um verschiedene Einrückungsebenen}
}
\newglossaryentry{st}
{
	name=Smart Tab,
	plural=Smart Tabs,
	description={Smart Tabs sind dazu gedacht, bei einem Zeilenwechsel die neue Zeile genauso einzurücken wie die alte Zeile}
}
%%% Beginn Glossareinträge

% Syntax:
%\newglossaryentry{label}
%{
%	name=Name,
%	plural=Namen,
%	description={Beschreibung}
%}

% Verwendung der Glossareinträge:
% Normalerweise \gls{label}
% Am Anfang des Satzes \Gls{label}
% Bei Plural: \glspl{label}
% Bei Plural am Anfang des Satzes: \Glspl{label}
% Falls nichts davon passt: \glslink{label}{Anderer Text}

%%% Ende Glossareinträge

\begin{document}
\maketitle

\section{Einleitung}


\pagebreak
\section{Kriterien}
% Diese Section sollte kurz und knapp "für Manager" sein
% und auf eine Seite passen.

\subsection{Muss}
\criterium{Benutzereingabe und Auswertung}{CMio}
Die Anwendung muss in der Lage sein, vom Nutzer eingegebene \glspl{lt} in \gls{vls} entgegen zu nehmen, diese nach der \gls{nr} auszuwerten und das Ergebnis in Textform separat auszugeben. Der Benutzer muss in der Lage sein, die Auswertung zu unterbrechen.
\criterium{Fehlerhandling}{CMerr}
Die Anwendung muss in der Lage sein, syntaktisch falsche Eingaben als solche zu erkennen und eine Fehlermeldung auszugeben.

% Syntax:
% \criterium{Überschrift des Kriteriums}{ctr:label}
% Beschreibung des Kriteriums

\subsection{Kann}
\criteriumOptional{Online-Funktionen}{COweb}
Die Anwendung kann über das Internet mittels \gls{brow} verwendet werden.
Dabei ist es möglich über vordefinierte Links Übungsaufgaben aufzurufen. Weiter kann aus der aktuellen Eingabe ein Link generiert werden, dessen Aufruf die Eingabe wiederherstellt.

\criteriumOptional{Editor}{COedit}
Anwender sollen ihre \glspl{lt} über einen Editor eingeben können. Dieser Editor soll einen Wechsel der Schriftgröße ermöglichen. Zusätzlich soll der Editor \glspl{mfe} und \glspl{st} unterstützen. Weiter soll der Nutzer eigene Kommentare schreiben können, die bei der Auswertung nicht berücksichtigt werden.

\criteriumOptional{Ausgabeformate}{COout}
Dem Nutzer sollen folgende Ausgabeformate zur Verfügung gestellt werden : 
\begin{itemize}
\item \gls{vls}
\item Unicode
\item LaTeX
\item Haskell-Code
\item Lisp-Code
\end{itemize}

\criteriumOptional{Reduktion}{COro}
Der Nutzer kann für eingegebene \glspl{lt} bestimmen, nach welcher \gls{aws} diese ausgewertet werden sollen. Dabei stehen folgende \glspl{aws} zur Verfügung:
\begin{itemize}
\item \gls{nr}
\item \gls{vbr}
\item \gls{cbn}
\item \gls{cbv}
\item \gls{ao}
\end{itemize}
Bei der Auswertung gibt es die Möglichkeiten einer \emph{Sofortauswertung}, bei der nur das Ergebnis angezeigt wird, oder einer \emph{Schritt-für-Schritt Auswertung}, bei der der Nutzer selbst den nächsten auszuwertenden Term bestimmt oder die Auswahl der \gls{aws} überlässt. Weiterhin soll es möglich sein einzelne Schritte rückgängig zu machen und alle bisher gemachten Schritte anzeigen zu lassen.

\criteriumOptional{Darstellungsformate}{COrep}
Dem Nutzer sollen im Editor folgende Darstellungsformate zur Verfügung stehen : 
\begin{itemize}
\item \gls{vls}
\item Unicode
\item Darstellung als Baum
\end{itemize}

\criteriumOptional{Fehlerbehandlung}{COerr}
Treten Fehler bei der Reduktion auf, so sind folgende Daten dem Nutzer anzuzeigen : 
\begin{itemize}
\item Zeilen- und Spaltenindex des Fehlers
\item das Zeichen, das den Fehler verursacht hat
\item die Art des Fehlers
\item Vorschläge und/oder Ansätze zur Fehlerbehebung
\end{itemize}
Weiterhin soll die Fehlerstelle im Editor markiert werden.

\criteriumOptional{Bibliotheken}{COlib}
Die Anwendung soll durch verschiedene Standardbibliotheken ergänzt werden. Diese umfassen : 
\begin{itemize}
\item natürliche Zahlen
\item arithmetische Operationen auf Zahlen
\item Tupel
\item Listen
\item Übungsaufgaben
\end{itemize}

% Syntax:
% \criteriumOptional{Überschrift des Kriteriums}{ctr:label}
% Beschreibung des Kriteriums


\subsection{Abgrenzung}

% Syntax:
% \criteriumNot{Überschrift der Abgrenzung}{ctr:label}
\criteriumNot{Nutzung über Mobilgeräte}{CNmobile}
Die Nutzung der Anwendung über mobile Endgeräte ist nicht vorgesehen und wird daher weder unterstützt noch getestet.

\pagebreak
%%%%%%%%%%%%%%
\section{Produkteinsatz}

\section{Produktumgebung}

%%%%%%%%%%%
\section{Funktionale Anforderungen}

% Syntax:
% \functionality{Überschrift der FA}{fnc:label}
% \fulfills{crt:label1}
% \fulfills{crt:label2}
% Beschreibung der FA

%%%%%%%%%%%
\section{Nicht-Funktionale Anforderungen}

% Syntax:
% \nonFunctionality{Überschrift der NA}{nfc:label}
% Brschreibung der NA

%%%%%%%%%%%
\section{Tests}

% Syntax:
% \test{Überschrift des Tests}{tst:label}
% \tests{fnc:label}
% \tests{nfc:label}
%
% \teststep{Stand}
% {Aktion}
% {Reaktion}
%
% \teststep{Stand}
% {Aktion}
% {Reaktion}
%
% ...

%%%%%%%%%%%%%
\pagebreak
\appendix

\section{Seitenentwürfe}

% Hier ganz normale figures mit Bildern

\section{Glossar}

\printglossaries

\end{document}
