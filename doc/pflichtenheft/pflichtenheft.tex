\documentclass[parskip=full,11pt,twoside]{scrartcl}

\usepackage[l2tabu,orthodox]{nag}
\usepackage[utf8]{inputenc}

\title{Wavelength $\bm{\lambda}$-IDE}
\author{Muhammet Guemues, Markus Himmel, Marc Huisinga,\\Philip Klemens, Julia Schmid, Jean-Pierre von der Heydt}

% section numbers in margins:
\renewcommand\sectionlinesformat[4]{\makebox[0pt][r]{#3}#4}

% header & footer
\usepackage{scrlayer-scrpage}
\lofoot{\today}
\refoot{\today}
\pagestyle{scrheadings}

%\usepackage[sfdefault,light]{roboto}
\usepackage[T1]{fontenc}
\usepackage[ngerman]{babel}
\usepackage{datetime2}
\usepackage{hyperref}
\usepackage[nameinlink]{cleveref}
\crefname{figure}{Abb}{Abb}
\usepackage[section]{placeins}
\usepackage{xcolor}
\usepackage{graphicx}
\hypersetup{
	pdftitle={Pflichtenheft},
	bookmarks=true,
}
\usepackage{csquotes}

\usepackage{amsmath}
\usepackage{bm}

\usepackage{glossaries}
\GlsSetQuote{+}
%
\usepackage{pkg/pflichtenheft}

\MakeOuterQuote{"}

\makeglossaries

%%% Beginn Glossareinträge

\newglossaryentry{lk}
{
	name={$\lambda$-Kalk{\"u}l},
	description={Ein formaler Kalkül, in dem Berechnungen nur auf Basis von
		Funktionsanwendung realisiert werden}
}

\newglossaryentry{lt}
{
	name={$\lambda$-Term},
	description={Ein syntaktisch gültiger Term im \gls{lk}},
	plural={$\lambda$-Terme}
}

\newglossaryentry{vls}
{
	name={vereinfachte $\lambda$-Syntax},
	description={Variante der herkömmlichen Syntax des \glslink{lk}{$\lambda$-Kalküls},
	bei der das $\lambda$-Symbol durch ein \texttt{\textbackslash}-Symbol ersetzt wird}
}

% Syntax:
%\newglossaryentry{label}
%{
%	name=Name,
%	plural=Namen,
%	description={Beschreibung}
%}

% Verwendung der Glossareinträge:
% Normalerweise \gls{label}
% Am Anfang des Satzes \Gls{label}
% Bei Plural: \glspl{label}
% Bei Plural am Anfang des Satzes: \Glspl{label}
% Falls nichts davon passt: \glslink{label}{Anderer Text}

%%% Ende Glossareinträge

\begin{document}
\maketitle

\section{Einleitung}


\pagebreak
\section{Kriterien}
% Diese Section sollte kurz und knapp "für Manager" sein
% und auf eine Seite passen.

\subsection{Muss}

\criterium{Eingabe von \glslink{lt}{$\lambda$-Termen}}{crt:input}
\glspl{lt} können in Form der \glslink{vls}{vereinfachten $\lambda$-Syntax}
mit der Tastatur in die Software eingegeben werden.

\criterium{Reduktion von \glslink{lt}{$\lambda$-Termen}}{crt:reduce}
Eingegebene \glspl{lt} können mit Hilfe der Normal-Reduktionsordnung vollständig
reduziert werden. Die so bestimmte Normalform kann in \glslink{vls}{vereinfachter $\lambda$-Syntax}
ausgegeben werden.

\criterium{Fehlermeldung bei invalider Eingabe}{crt:error}
Beim dem Versuch, einen syntaktisch inkorrekten \gls{lt} reduzieren zu lassen, wird eine
Fehlermeldung ausgegeben.

\criterium{Abbruch der Ausführung}{ctr:cancel}
Der Benutzer kann die Reduktion eines \glslink{lt}{$\lambda$-Terms} abbrechen.

% Syntax:
% \criterium{Überschrift des Kriteriums}{crt:label}
% Beschreibung des Kriteriums

\subsection{Kann}

\criteriumOptional{Verfügbarkeit im Internet}{crt:webapp}
Die Anwendung kann ohne Installation im Internetbrowser ausgeführt werden.

\criteriumOptional{Weitere Reduktionsordnungen}{crt:reducestrats}
Die Anwendung kann neben der Normal-Reduktionsordnung auch \enquote{Call-by-name}
und \enquote{Call-by-value} als Reduktionsstrategie verwenden.

\criteriumOptional{Weitere Ausgabeformate}{crt:output}
Die eingegebenen und reduzierten \glspl{lt} können neben der \glslink{vls}{vereinfachten $\lambda$-Syntax}
auch als Unicode-Text, \LaTeX-Quellcode, Haskell-Quellcode und Lisp-Quellcode formatiert und ausgegeben
werden.

\criteriumOptional{Darstellungsformate}{crt:display}
Die eingegebenen und reduzierten \glspl{lt} können innerhalb der Applikation in
\glslink{vls}{vereinfachter $\lambda$-Syntax}, Unicode-Darstellung und als Syntaxbaum
dargestellt werden.

\criteriumOptional{Ausgabe von Teilschritten}{crt:partial}
Die zum Erreichen der Normalform notwendigen Reduktionsschritte gemäß der ausgewählten
Reduktionsstrategie können im ausgewählten Ausgabeformat ausgegeben werden.

% Syntax:
% \criteriumOptional{Überschrift des Kriteriums}{crt:label}
% Beschreibung des Kriteriums


\subsection{Abgrenzung}

% Syntax:
% \criteriumNot{Überschrift der Abgrenzung}{crt:label}

\pagebreak
%%%%%%%%%%%%%%
\section{Produkteinsatz}

\section{Produktumgebung}

%%%%%%%%%%%
\section{Funktionale Anforderungen}

% Syntax:
% \functionality{Überschrift der FA}{fnc:label}
% \fulfills{crt:label1}
% \fulfills{crt:label2}
% Beschreibung der FA

%%%%%%%%%%%
\section{Nicht-Funktionale Anforderungen}

% Syntax:
% \nonFunctionality{Überschrift der NA}{nfc:label}
% Brschreibung der NA

%%%%%%%%%%%
\section{Tests}

% Syntax:
% \test{Überschrift des Tests}{tst:label}
% \tests{fnc:label}
% \tests{nfc:label}
%
% \teststep{Stand}
% {Aktion}
% {Reaktion}
%
% \teststep{Stand}
% {Aktion}
% {Reaktion}
%
% ...

%%%%%%%%%%%%%
\pagebreak
\appendix

\section{Seitenentwürfe}

% Hier ganz normale figures mit Bildern

\section{Glossar}

\glsaddall
\printglossaries

\end{document}
