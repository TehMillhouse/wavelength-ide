\documentclass[parskip=full,11pt,twoside]{scrartcl}

\usepackage[l2tabu,orthodox]{nag}
\usepackage[utf8]{inputenc}

\title{Wavelength $\bm{\lambda}$-IDE}
\author{Muhammet Guemues, Markus Himmel, Marc Huisinga,\\Philip Klemens, Julia Schmid, Jean-Pierre von der Heydt}

% section numbers in margins:
\renewcommand\sectionlinesformat[4]{\makebox[0pt][r]{#3}#4}

% header & footer
\usepackage{scrlayer-scrpage}
\lofoot{\today}
\refoot{\today}
\pagestyle{scrheadings}

\usepackage[sfdefault,light]{roboto}
\usepackage[T1]{fontenc}
\usepackage[ngerman]{babel}
\usepackage{datetime2}
\usepackage{hyperref}
\usepackage[nameinlink]{cleveref}
\crefname{figure}{Abb}{Abb}
\usepackage[section]{placeins}
\usepackage{xcolor}
\usepackage{graphicx}
\hypersetup{
	pdftitle={Pflichtenheft},
	bookmarks=true,
}
\usepackage{csquotes}
\MakeOuterQuote{"}

\usepackage{amsmath}
\usepackage{bm}

\usepackage{pkg/pflichtenheft}

\begin{document}
\maketitle

\section{Einleitung}


\pagebreak
\section{Kriterien}
% Diese Section sollte kurz und knapp "für Manager" sein
% und auf eine Seite passen.

\subsection{Muss}

% Syntax:
% \criterium{Überschrift des Kriteriums}{ctr:label}
% Beschreibung des Kriteriums

\subsection{Kann}

% Syntax:
% \criteriumOptional{Überschrift des Kriteriums}{ctr:label}
% Beschreibung des Kriteriums


\subsection{Abgrenzung}

% Syntax:
% \criteriumNot{Überschrift der Abgrenzung}{ctr:label}

\pagebreak
%%%%%%%%%%%%%%
\section{Produkteinsatz}

\section{Produktumgebung}

%%%%%%%%%%%
\section{Funktionale Anforderungen}

% Syntax:
% \functionality{Überschrift der FA}{fnc:label}
% \fulfills{crt:label1}
% \fulfills{crt:label2}
% Beschreibung der FA

%%%%%%%%%%%
\section{Nicht-Funktionale Anforderungen}

% Syntax:
% \nonFunctionality{Überschrift der NA}{nfc:label}
% Brschreibung der NA

%%%%%%%%%%%
\section{Tests}

% Syntax:
% \test{Überschrift des Tests}{tst:label}
% \tests{fnc:label}
% \tests{nfc:label}
%
% \teststep{Stand}
% {Aktion}
% {Reaktion}
%
% \teststep{Stand}
% {Aktion}
% {Reaktion}
%
% ...

%%%%%%%%%%%%%
\pagebreak
\appendix

\section{Seitenentwürfe}

% Hier ganz normale figures mit Bildern

\section{Glossar}

% Syntax?
% \textbf{Eintrag}:
% Erklärung

\end{document}
