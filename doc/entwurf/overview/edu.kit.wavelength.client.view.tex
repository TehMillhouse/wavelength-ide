The \texttt{\pkglnk{view}} package manages general aspects of the application.
Among these are building the GUI, managing the interaction between the Model and the View and serialization.

The \lnk{URLSerializer} class manages the serialization of the application's state. 
Hence it needs the objects that are serialized and the recipients of the serialization.
The recipients must implement the \lnk{SerializationObserver} interface in order to update the serialization.
Note that the \lnk{URLSerializer} needs a polling delay time in milliseconds for concurrency.

The \lnk{App} class builds the GUI components and the associated Actions. 
It also stores an instance of the \texttt{\hyperref[type:edu.kit.wavelength.client.view.execution.Executor]} class.
Note that the \lnk{App} class works similar to a Singleton by guaranteeing that only one instance exists and is accessed using a static method.
Hence each \texttt{\hyperref[type:edu.kit.wavelength.client.view.action.Action]} can access the \lnk{App} 
in order to use the Model via the \texttt{\hyperref[type:edu.kit.wavelength.client.view.execution.Executor]} class
or get UI components by calling the respective methods.